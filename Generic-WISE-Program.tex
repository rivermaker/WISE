\documentclass[12pt]{beamer}
\title[WISE]{Women In Software Engineering}
\subtitle{A TalentSprint Initiative}
\author[WISE]{Asokan Pichai}
\institute[TalentSprint]{TalentSprint}
\date{Licensed To Skill}
\usetheme{CambridgeUS}

\begin{document}
{\large
\maketitle

\begin{frame}{Structure of Document}
  \tableofcontents
\end{frame}

\section{Overview}
\begin{frame}{Genesis}
  \begin{itemize}
	  \item Software Product Companies (almost exclusively foreign) are concerned about high gender imbalance in the workplace.
  \item These organizations have programs and initiatives to address this imbalance and bring about the necessary diversity and inclusion.
  \item WISE program offers a way to reach out to these organizations and bring them to the few enlightened colleges that can address this issue for them.
  \end{itemize}
\end{frame}  

\begin{frame}{Vision for Women Students}  
  \begin{itemize}
  \item Impart higher order software engineering skills.
  \item Make them understand what it means to work with large codebases (20 - 100 kLOC).
  \item Build a complement of skills and knowledge to make them confident and competent.
  \item Develop an impressive resume with visible and verifiable expertise and experience.
  \item Make them recruitable by large multi-national Software Product Organizations.
  \end{itemize}
\end{frame}

\begin{frame}{Overview}
  \begin{description}[DURATION]
  \item[Audience] Students in semester 2
  \item[Duration] 300+ hours over semesters 2,3,4,5,6 
  \item[Mode] Instruction using Experiential Learning and Summer Projects
  \item[Delivery] ILT \& Remote :: Using IPEARL 
  \end{description}
\end{frame}

\section{Learning Outcomes}
\begin{frame}{Learning Outcomes and Competencies}
  At the end of the program, the successful participants will be able to:
  \begin{itemize}
  \item learn to learn new technologies and skills,
  \item acquire a strong knowledge of modern programming platforms and practices, 
  \item work on distributed development project teams,
  \item communicate and present technical ideas effectively,
  \item comfortably deal with large code-bases, and
  \item leverage the above to succeed in campus selection.
  \end{itemize}
\end{frame}

\section{Coverage and Schedule}
\begin{frame}{Lecture Modules - Overview}
    \begin{itemize}
    \item There are five Lecture Modules.
    \item These take place during the semester.
    \item Duration: 30 hours each. 
    \item Delivered using TalentSprint's proven hands-on, experiential methodology.
    \end{itemize}
\end{frame}

\begin{frame}{Lecture Modules - for Circuit Branches} 
  \begin{description}[WWW]
  \item[PCP] 1.2 Professional C Programming
    \begin{itemize}
      \item Linux, gcc, vim
    \end{itemize}
  \item[BWA] 2.1 Building simple Web Applications
    \begin{itemize}
      \item HTML + CSS, Bootstrap, JS + JQuery, SQL
    \end{itemize}
  \item[EJD] 2.2 Enterprise Java Development
    \begin{itemize}
    \item Java, git, junit 
    \end{itemize}
  \item[BMA] 3.1 Building Mobile Applications
    \begin{itemize}
      \item Adv. Bootstrap, PhoneGap
    \end{itemize}
  \item[BDC] 3.2 Big Data and the Cloud 
    \begin{itemize}
      \item Python, Spark 
    \end{itemize}
  \end{description}
\end{frame}

\begin{frame}{Lecture Modules - for non-Circuit Branches} 
  \begin{description}[WWW]
  \item[PCP] 1.2 Professional C Programming
    \begin{itemize}
      \item Linux, gcc, vim
    \end{itemize}
  \item[BWA] 2.1 Building Web Applications
    \begin{itemize}
      \item HTML, CSS, JS, JQuery, BootstrapSQL
    \end{itemize}
  \item[EJD] 2.2 Enterprise Java Development 
    \begin{itemize}
    \item Java, git, junit 
    \end{itemize}
  \item[ISC] 3.1 Introductory Scientific Computing 
    \begin{itemize}
      \item SciPy/SciLab, MatplotLib: Core Libraries
    \end{itemize}
  \item[ASC] 3.2 Advanced Scientific Computing
    \begin{itemize}
      \item SciPy/SciLab Advanced Libraries
    \end{itemize}
  \end{description}
\end{frame}

\begin{frame}{Coverage: Project Modules}
  \begin{itemize}
  \item There are three Project Modules. 
  \item They are covered in summer. 
  \item Duration: 50 hours each.
  \item Focus is on exploration, learning to learn, applying knowledge, processes, methodologies, and team work.
  \end{itemize}
\end{frame}

\begin{frame}{Coverage: Project Modules}
  \begin{description}[WWW]
  \item[MPP] 1.S Modern Programming Practice
    \begin{itemize}
    \item Python, pyUnit
    \end{itemize} 
  \item[JWD] 2.S Java Web Devleopment
    \begin{itemize}
    \item Adv. Java, JSP, Servlets
    \end{itemize} 
  \item[MPW] 3.S Mini-Project Work or Internship
  \end{description}
\end{frame}

\begin{frame}{Sequence and Duration}
  \begin{tabular}{|l|l|l|r|} \hline
    Module & When   & Elapsed  & Hours \\
           &        & Time     &       \\\hline\hline
    PCP    & Sem 2  & Semester & 30    \\\hline
    \alert{MPP}   & \alert{2 -- 3}& \alert{2 weeks} & \alert{50}   \\\hline
    BWA    & Sem 3  & Semester & 30    \\\hline
    EJD   & Sem 4  & Semester & 30    \\\hline
    \alert{JWD}   & \alert{4 -- 5}& \alert{2 weeks} & \alert{50}    \\\hline
    DMA/ISC    & Sem 5  & Semester & 30    \\\hline
    BDC/ASC    & Sem 6  & Semester & 30    \\\hline
    \alert{MPW}   & \alert{6 -- 7}& \alert{2 weeks} & \alert{50}   \\\hline\hline
    \textbf{Total} &        &          & \textbf{300}  \\\hline\hline
  \end{tabular}
\end{frame}

\section{Staffing and Logistics}
\begin{frame}{Faculty and Logistics}
  \begin{itemize}
  \item Main faculty will travel to build rapport with students in different batches, taking classes for about a week.
  \item The class/location where main faculty is present will be the \emph{studio}.
  \item Each class will have 1/2 Junior faculty.
  \item In addition there will be sysadmin/MOODLE administrator.
  \end{itemize}
\end{frame}

\begin{frame}{Infrastructure}
  \begin{itemize}
  \item Each student to have a computer in the class, with a headphone/microphone.
  \item Each class to have two projectors, webcam and connectivity.
  \item Each class to have a server/dias Computer.
  \item Two central servers at each location to be provided.
\end{itemize}  
\end{frame}

\begin{frame}{Infrastructure}
\begin{itemize} 
  \item Standard software such as IDEs, compilers, iTalc, documentation etc., to be installed in each machine.
    \item TalentSprint will provide an exact list and also assist in setting up the systems and servers. 
    \item TalentSprint will train the support staff, sysadmins and Junior faculty for handling these.
  \end{itemize}
\end{frame}

\section{Philosophy}
\begin{frame}{Meritocracy}
	At the end of every module, we drop the bottom 10% 
\end{frame}
\section{Story So Far}
\begin{frame}{Current Partners}
	\begin{itemize}
		\item BVRITH -- First batch completed in 2015 June
		\item SVECW -- First Batch will complete 2016 Summer
		\item Top 10 students of BVRIT Batch 1 picked up for mentoring by Microsoft
		\item 30 students of Batch 1 landed internships at the end of their third year -- in ADP, Pega, Virtusa and two start-ups: Dana Systems, Thrymr
		\item Top 30 presenters from BVRIT Batch 2 presented before industry experts recently. They have been asked to present before the seniors in their companies.
	\end{itemize}
\end{frame}

\section{Detailed Module Coverage}
\begin{frame}{PCP: Professional C Programming}
  \begin{description}[Resource(S)]
  \item[Slot] 1.2 
  \item[Platform] Linux, gcc
  \item[Objectives] On successful completion participants will be able to: reinforce their learning of C, read code, write good procedural code.
  \item[Pedagogy] Solving a series of graduated problems.
  \item[Assessment] Assignments, quizzes and final test.
  \item[Resource(s)] TalentSprint Problem Bank, Project Euler
  \end{description}
\end{frame}

\begin{frame}{MPP: Modern Programming Practice}
  \begin{description}[Resource(S)]
  \item[Slot] 1.S
  \item[Platform] Python, pyUnit
  \item[Objectives] On successful completion participants will be able to: learn to learn, identify relevant libraries, read documentation, write good procedural code, write unit tests and build small projects.
  \item[Pedagogy] Solving a series of graduated problems followed by group project.
  \item[Assessment] Assignments, final test and presentations.
  \item[Resource(s)] TalentSprint Problem Bank, Project Euler
  \end{description}
\end{frame}

\begin{frame}{Building Web Applications}
  \begin{description}
  \item[Slot] 2.1
  \item[Platform] HTML, CSS, JS, JQuery, Bootstrap, SQL
  \item[Objectives] On successful completion participants will be able to: build simple web sites and applications, using existing databases
  \item[Pedagogy] Constructing  web applications/sites in teams
  \item[Assessment] In class assignments, quizzes and tests.
  \item[Resource(s)] TalentSprint Problem Bank
  \end{description}
\end{frame}

\begin{frame}{EJD: Enterprise Java Development}
  \begin{description}
    \item[Slot] 2.2
  \item[Platform] Java, git, jUnit \ldots
  \item[Objectives] On successful completion participants will be able to: understand the object paradigm, use version control, bug-tracking tools, work with large body of code and tests, and understand FOSS development processes and practices.
  \item[Pedagogy] Projects
  \item[Assessment] In class assignments, Presentations.
  \item[Resource(s)] TalentSprint Project/Problem Bank
  \end{description}
\end{frame}

\begin{frame}{JWD: Java Web Devlopment}
  \begin{description}
    \item[Slot] 2.S
  \item[Platform] Adv. Java, JSP, Servlets 
  \item[Objectives] On successful completion participants will be able to: work with teams, read large body of code, use build and test systems,  write web applications using the Java ecosystem
  \item[Pedagogy] Group Projects
  \item[Assessment] Evaluation by External People
  \item[Resource(s)] TalentSprint Project Bank
  \end{description}
\end{frame}

\begin{frame}{DMA: Developing Mobile Applications}
  \begin{description}
   \item[Slot] 3.1
  \item[Platform] Adv. Bootstrap, PhoneGap
  \item[Objectives] On successful completion of the module, participants will be able to: build applications that run on multiple mobile platforms (Android, iOS)
  \item[Pedagogy] Working in groups on assigned applications.
  \item[Assessment] Project Evaluation.
  \item[Resource(s)] TalentSprint Project Bank
  \end{description}
\end{frame}

\begin{frame}{BDC: Big Data and the Cloud}
  \begin{description}
    \item[Slot] 3.2
  \item[Platform] Python
  \item[Objectives] On successful completion of the module, participants will be able to: use APIs of Google, FaceBook, Amazon, use Spark, MapReduce and equivalents work in teams, perform Analytics on Big Data.
  \item[Pedagogy] Students will work in groups on assigned tasks.
  \item[Assessment] Projects and presentations.
  \item[Resource(s)] TalentSprint Problem Bank
  \end{description}
\end{frame}

\begin{frame}{ISC: Introductory Scientific Computing}
  \begin{description}
   \item[Slot] 3.1 
  \item[Platform] SciPy/SciLab, Matplotlib
  \item[Objectives] On successful completion of the module, participants will be able to: interactively solve their academic problems and assignments using the tools, producing charts etc.,.
  \item[Pedagogy] Working as individuals in solving/redoing academic assignments.
  \item[Assessment] End-of-module tests.
  \item[Resource(s)] TalentSprint Problem Bank, FOSSEE TextBook Project
  \end{description}
\end{frame}

\begin{frame}{ASC: Advanced Scientific Computing}
  \begin{description}
    \item[Slot] 3.2
  \item[Platform] SciPy/SciLab
  \item[Objectives] On successful completion of the module, participants will be able to: use advanced libraries and write programs that solve larger problems in their academic stream.
  \item[Pedagogy] Students will work in groups on assigned tasks.
  \item[Assessment] Projects and presentations.
  \item[Resource(s)] TalentSprint Problem Bank, FOSSEE TextBook Project
  \end{description}
\end{frame}

\begin{frame}{MPW: Mini-Project Work -- Option 1}
  \begin{description}[Resource(S)]
    \item[Slot] 3.S
  \item[Platform] Python/Java/C on Arduino or Raspberry Pi
  \item[Objectives] On successful completion of the module, participants will be able to: work in inter-disciplinary teams, plan, allocate and integrate their work, debug in real-world situations, work on hardware/software combo.
  \item[Pedagogy] Students will work as a team to select and implement a cross-domain project.
  \item[Assessment] Presentation and demonstration.
  \item[Resource(s)] Arduino, Raspberry Pi forums.
  \end{description}
\end{frame}

\begin{frame}{MPW: 3.S: Mini-Project Work -- Option 2}
  \begin{description}
  \item[Platform] Python or Java or C 
  \item[Objectives] On successful completion of the module, participants will be able to: plan, allocate tasks and integrate their work, test and debug in real-world situations,prioritize and rescope their tasks, program for SMAC.
  \item[Pedagogy] Students will work as a team to select and implement a SMAC project.
  \item[Assessment] Presentation and demonstration.
  \item[Resource(s)] FB, Google \ldots developer forums
  \end{description}
\end{frame}

}
\end{document}

